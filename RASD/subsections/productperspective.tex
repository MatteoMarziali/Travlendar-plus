Our idea is to create a personal companion application to help users managing and organizing their daily life. According to this intention, we would like to realize an extremely friendly user interface and a lightweight software in order to make Travlendar+ affordable to many people and runnable by many devices.
In order to make use of every functionality the devices require GPS service and an internet connections for most of the services.
Since Travlendar+ is going to offer many routes depending on different travel means, it will necessarily have to interact with many institutions such as public transport and car/bike sharing providers. This aspect will affect both the software and the hardware design. Indeed, it is necessary to query data about the shared cars, bikes and the taxis location around the city and to retrieve information about trains and buses schedules. 
Hence, our system must be very fast and dynamic to support a huge number of query in a few seconds, moreover to interview external databases it's strictly required that the users have an active internet connection. 
Concerning the hardware, we intend to have a database which only contains username and password of all our customers. For sure, it seems useless having a database for those kinds of data but in our idea this choice allows the app to be updated and improved easily in the future, for instance saving on the database clients' routes and meetings. 

