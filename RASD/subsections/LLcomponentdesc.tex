\begin{itemize}
	\item \textbf{[\hypertarget{Meeting manager bean}{Meeting manager bean}]}  This is the component responsible for managing the insertion of meeting data by the user, it also provides these details to the meeting evaluator bean and to conflict evaluator bean.
	
	\item \textbf{[\hypertarget{Meeting evaluator bean}{Meeting evaluator bean}]} The meeting evaluator bean has the role to evaluate if a meeting can be scheduled with respect to the adjacent meetings and the breaks. To execute this critical computation it needs information about the breaks, the new meeting and the time to reach the meetings which precede and follow it. The result of the meeting evaluator bean is required by the conflict evaluator to establish whether it is necessary to generate a conflict or not.  \label{G3}
	
	\item \textbf{[\hypertarget{Warning manager bean}{Warning manager bean}]} This component represents the bean which is responsible to generate a warning and submit it to the user if the meeting evaluator notify a conflict.  \label{G4}
	
	\item \textbf{[\hypertarget{Conflict solver bean}{Conflict solver bean}]} The conflict solver is the component that manages the resolution of conflicts, thus it is called by a new warning and his functions are to ask the user whether ignore or modify the meetings involved in a conflict and to apply his choice. \label{G5}
	
	\item \textbf{[\hypertarget{Signup manager bean}{Signup manager bean}]} This is the component which manages the user signup process, from the insertion of personal data, through their storage on the database, to the account verification.  \label{G6}
	
	\item \textbf{[\hypertarget{Login manager bean}{Login manager bean}]} This is the component dedicated to all the login steps. When the user insert in the apposite area his name and password, this component takes the data and provides the computations needed to verify the credentials and load the user's content . \label{G7}
	
	\item \textbf{[\hypertarget{Preferences manager bean}{Preferences manager bean}]} This is the component designed to assolve all the functions related to the preferences. Hence, it manages the insertion, update and deletion of the preferences about routes, travel means and breaks, furthermore it provides preferences details to other components when it is necessary. \label{G8}
		
	\item \textbf{[\hypertarget{Route calculator bean}{Route calculator bean}]} The route calculator is a session bean that deals with everything concerns travels computations. It retrieves information such as weather forecasts, public transports news and maps data from external servers matching the user preferences and provides the optimal route.  \label{G7}
			
\end{{temize}

