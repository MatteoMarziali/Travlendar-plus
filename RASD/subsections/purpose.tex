The main purpose of Travlendar+ is to create a software that allows users to easily manage their daily meetings and commitments, by providing some useful features such as finding the best means of transport to reach the appointment place and easily know the quickest route available to be punctual.
Specifically, our goals consist in:
\begin{itemize}

\item[G1] Providing a calendar and the possibility to memorize events and appointments on it.

\item[G2] Automatically computing and accounting for travel time between appointments to make sure that the user is not late for them.


\item[G3] Automatically generating warnings to notify the user that at least two meetings are overlapping (this means the timings are incompatible, in other words the user can't reach in time the second meeting if the first meeting finishes on time).

\item[G4] Providing routes and travels according to user preferences about the preferred/prohibited travel means and daily breaks set.

\item[G5] Possibility to add reminders for a meeting in order to prevent the users forgetting their appointments.

\item[G6] Allowing the users to modify appointment schedules.

\item[G7] Automatically notifying people involved in a specific meeting if the user is late and has selected this feature previously. 


\end{itemize}

On the other hand, the purpose of this paper is to define in a detailed way all the functions and requirements of our application. In doing this, we start focusing on a brief overview to characterize the product with relevance to its interaction with the world, then we will proceed deeply in analysing which functions are relevant and should be provided, and which requirements are needed to the stakeholders. 
