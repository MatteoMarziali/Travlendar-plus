
\begin{itemize}

\item \textbf{[A1] Signup and Login}

Considering that the assignments provided do not say much generally about users without any reference to a possible signup or login, we assume that the registration is mandatory to create the first meeting, then every access to the app requires the login to manage each event saved. Please note that login parameters could be memorized to save and recover easily a user instance.

\item \textbf{[A2] Meeting management}

According to the requirements, we want to develop a system which allows the user to set his preferences with regards to the travels. Moreover, we decide that he can also cancel or anticipate/postpose an event, assuming a previous reschedule agreement among the participants. It goes without saying that an appointment can also be modified, this means that a user can change either the starting location or the arrival location, the hour, the date and the other details chosen during the creation of the meeting, always making the same rescheduling agreement assumption. 

\item \textbf{[A3] Warnings}

Our assumptions about the warning are the following: when the system generates a warning, the app allows the user to modify the related event that could be cancelled or delayed. In case of the user postposes the meeting, if he provided the email addresses of the other people involved in the appointment, Travlendar+ automatically will notify them that a change occurs. 

\item \textbf{[A4] Routes}

Concerning the routes, we decided to manage them in this way. 
The system generates different routes according to the user preferences, it will be the user itself to decide which itinerary fits better with him among the alternatives. 

\item \textbf{[A5] Preferences}

As far as the preferences are concerned, we decided that they belong to a user instead of a meeting. This means that a user cannot define different preferences for each meeting, while they are valid for every appointment.

\end{itemize}

 
