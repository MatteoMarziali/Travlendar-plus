This document is structured in three parts:
\begin{itemize}
	\item \textbf{Introduction}: In the first introductory section, we give a short description both of the goals and of the environment which our app has to deal with. Moreover, we explain some notes useful to understand and read the whole paper. 
	\item \textbf{Overall Description}: gives a general description of the application, focusing on the context of the system, going in details about domain assumptions and constraints. The aim of this section is to provide a context to the whole project and show its integration with the real world and showing the possible interactions between the user, the system and the world itself. 
	\item \textbf{Specific Requirements}: this section contains all of the software requirements to a level of detail aimed to be enough to design a system to satisfy said requirements, and testers to test that the system actually satisfies them. It also contains the detailed description of the possible interactions between the system and the world with a simulation and preview of the expected response of the system with given stimulation. 
\\Finally, we express the requirements through the Alloy model, which allows us to define the interactions, the functions and the constraints that characterize Travlendar+ using a formal language.\\
The document ends with a short note about the effort spent in producing it and at last you can also find useful references.
\end{itemize}