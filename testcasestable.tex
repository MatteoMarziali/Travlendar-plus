\subsection{Login/Signup}

\begin{flushleft}

\begin{table}[htp]

\begin{tabular}{p{7cm}|p{7cm}}
 Implemented requirement&Notes\\
\hline
\hline
R1: Registration&\textcolor{green}{Accomplished}\\
\hline
R2: Login&\textcolor{green}{Accomplished}\\
\hline

\end{tabular}

\caption{Req table } 
\label{tab:Reqtable}

\end{table}

\end{flushleft}



\subsection{Event CRUD operations}

\begin{flushleft}
	
	\begin{table}[htp]
		
		\begin{tabular}{p{7cm}|p{7cm}}
			Implemented requirement&Notes\\
			\hline
			\hline

R3: The insertion of new event in the calendar. & \textcolor{green}{Accomplished}. See \hyperlink{meetingRemark}{Additional notes.}\\
\hline
R4: Modification of an event. & \textcolor{green}{Accomplished}\\
\hline
R5: Deletion of an event. & \textcolor{green}{Accomplished}\\
\hline
R7: Warning creation. & \textcolor{BurntOrange}{Partially accomplished} Everything works fine excepts for a minor bug. We made a test case consisting in inserting a meeting from 22.00p.m. to 23.00p.m. in Busto Arsizio and a meeting from 2.00a.m. to 3.00a.m. in Puglia on the following day, obviously the insertion should be prevented due to the impossibility to cover the distance between the meetings in just 3 hours, but this doesn't happen, the second meeting is accepted by the system.  \\
\hline

\end{tabular}

\caption{Event table } 
\label{tab:Eventtable}

\end{table}

\end{flushleft}



\subsection{Routes}

\begin{flushleft}
	
	\begin{table}[htp]
		
		\begin{tabular}{p{7cm}|p{7cm}}
			Implemented requirement&Notes\\
			\hline
			\hline
R8: The identification of the best mobility option for a transfer, according to user preferences and other external factors.& \textcolor{green}{  Accomplished} \\
\hline

\end{tabular}

\caption{Mobility table } 
\label{tab:Mobilitytable}

\end{table}

\end{flushleft}



\subsection{Breaks}

\begin{flushleft}
	
	\begin{table}[htp]
		
		\begin{tabular}{p{7cm}|p{7cm}}
			Implemented requirement&Notes\\
			\hline
			\hline
Definition of flexible daily breaks&\textcolor{green}{Accomplished}\\
\hline
Modification of flexible daily breaks&\\
\hline
Usage of flexible daily breaks&\\
\hline

\end{tabular}

\caption{Break table } 
\label{tab:Breaktable}

\end{table}

\end{flushleft}

\clearpage

\subsection{Travel mean preferences}

\begin{flushleft}
	
	\begin{table}[htp]
		
		\begin{tabular}{p{7cm}|p{7cm}}
			Implemented requirement&Notes\\
			\hline
			\hline
R14: Some settings regarding the navigation profile, such as choosing the preferred means of transport&\textcolor{BurntOrange}{Partially accomplished} The unique preference the user can manage regards the travel means and in particular a prevalence order of them. This leads to a low level of customization for the users. Moreover the prevalence order prevent a user avoiding to adopt a specific travel mean, indeed it is not excluded but at most it is placed below other transports.   \\
\hline

\end{tabular}

\caption{Nav table } 
\label{tab:Navtable}

\end{table}

\end{flushleft}