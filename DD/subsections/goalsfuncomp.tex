\begin{flushleft}

\begin{table}[htp]

\begin{tabular}{p{3cm}|p{6cm}|p{2cm}}
 Goal&Functions&Components\\

\hline
\hline
\textbf{[\hypertarget{G1}{G1}]}  Memorizing and organizing both events and appointments on the calendar. &\textbf{[\hypertarget{F2}{F2}] Meeting creation}: This is the most important function of the app, it allows to generate an event related to an appointment. It requires the user to define all the details such as date, time, location, starting point, preferences etc. & \autoref{tab:meetingmanagertable}\\
\hline
 \textbf{[\hypertarget{G2}{G2}]} Being sure to reach the appointment location in time avoiding delays. &
 \begin{itemize}
 	\item \textbf{[\hypertarget{F6}{F6}] Reminder management}: The applications allows users to set up reminders for a certain Meeting in order for the user not to be late for it.
 	
 	\item \textbf{[\hypertarget{F8}{F8}] Update meetings}: This function is both basic and relevant, it allows the user to customize his meetings after their creation. In other words, through this functionality the user can modify each one meeting details, even in case of a warning is generated.
 \end{itemize}
& 
\begin{itemize}
	\item \autoref{tab:routecalculatortable}
	
	\item \autoref{tab:meetingmanagertable}
\end{itemize}
\\
\hline
\textbf{[\hypertarget{G3}{G3}]} Being sure not to schedule overlapping meetings. &
\begin{itemize}
	\item \textbf{[\hypertarget{F9}{F9}] Warnings management}:  In case a warning is generated by the system due to a possible overlap among two or more meetings, the user must solve the warning. In other words, the user has to decide wheter he wants to ignore the overlap notification or he intend to modify some meetings to be sure that he can reach and participate to all his appointments.
	
	\item \textbf{[\hypertarget{F4}{F4}] Delays management}:  If the app had noticed, according to the estimated travel time, that the user is in late, and he previously had inserted the email address of the meeting’s participants, Travlendar+ would notify them about the delay. 
\end{itemize}&
\begin{itemize}
	\item \autoref{tab:conflictcheckertable}
	
	\item \autoref{tab:conflictsolvertable}
	
	\item \autoref{tab:warningmanagertable}
\end{itemize}\\
\hline


\end{tabular}
\caption{Traceability table } 
\label{tab:traceabilitytable1}
\end{table}

\begin{table}[htp]
	
\begin{tabular}{p{3cm}|p{6cm}|p{2cm}}
\hline
\textbf{[\hypertarget{G4}{G4}]} Being able to use only travel means that fit with user preferences. &\textbf{[\hypertarget{F3}{F3}] Preferences set up}: An important feature of Travlendar+ consists in allowing the user to filter out specific routes depending on some constraints about the travel, or to set break-dedicated time slots.&
\autoref{tab:preferencesmanagertable}\\
\hline
\textbf{[\hypertarget{G5}{G5}]} Preventing the user to forget an appointment.&\textbf{[\hypertarget{F7}{F7}] Recurrent events management}: The smartest function Travlendar+ will offer; it consists in allowing the user to select events to be rescheduled periodically just creating one meeting. Done this choice, the app. Automatically manages to  reschedule the specific meeting according to the period that the user establish, for instance one week, one month.& \autoref{tab:remindermanagertable}\\
\hline
\textbf{[\hypertarget{G6}{G6}]} Allowing the users to modify appointment schedules.&\textbf{[\hypertarget{F8}{F8}] Update meetings}: This function is both basic and relevant, it allows the user to customize his meetings after their creation. In other words, through this functionality the user can modify each one meeting details, even in case of a warning is generated.&
\autoref{tab:meetingmanagertable}\\
\hline
\textbf{[\hypertarget{G7}{G7}]} Notifying other people involved in a meeting about user eventual delay.&\textbf{[\hypertarget{F5}{F5}] Route generation}: the main hidden function of Travlendar+ is to automatically compute and suggest to the user the best travel among those which fit the preferences he has selected.& \autoref{tab:meetingmanagertable}\\
\hline
-&\textbf{[\hypertarget{F1}{F1}] Signup and Login}: Travlendar+ users must sign up the first time they intend to create a meeting and further usages of the app will require a login to access all its functionalities.&\begin{itemize}
	\item \autoref{tab:signupmanagertable}
	
	\item\autoref{tab:loginmanagertable}
\end{itemize}\\
\hline

\end{tabular}

\caption{Traceability table } 
\label{tab:traceabilitytable2}

\end{table}

\end{flushleft}