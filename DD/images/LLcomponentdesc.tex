\begin{flushleft}

\begin{table}[htp]

\begin{tabular}{l|l}
 Component&analysis\\
\hline
\hline
Name&Meeting manager bean\\
\hline
Input interfaces&\item Meeting data input \item Meeting evaluation\\
\hline
Output interfaces& Meeting details\\
\hline
Description&This is the component responsible for managing the insertion of meeting data by the user, it also provides these details to the meeting evaluator bean and to conflict evaluator bean.\\
\hline


\end{tabular}

\caption{Meeting manager  table } 
\label{tab:meetingmanagertable}

\end{table}

\end{flushleft}

\begin{flushleft}
	
	\begin{table}[htp]
		
		\begin{tabular}{l|l}
			Component&analysis\\
			\hline
			\hline
			Name&Meeting evaluator bean\\
			\hline
			Input interfaces&\item Meeting data  \item Break details \item Optimal route\\
			\hline
			Output interfaces& Meeting evaluation\\
			\hline
			Description&The meeting evaluator bean has the role to evaluate if a meeting can be scheduled with respect to the adjacent meetings and the breaks. To execute this critical computation it needs information about the breaks, the new meeting and the time to reach the meetings which precede and follow it. The result of the meeting evaluator bean is required by the conflict evaluator to establish whether it is necessary to generate a conflict or not. \\
			\hline
			
			
		\end{tabular}
		
		\caption{Meeting evaluator  table } 
		\label{tab:meetingevaluatortable}
		
	\end{table}
	
\end{flushleft}

\begin{flushleft}
	
	\begin{table}[htp]
		
		\begin{tabular}{l|l}
			Component&analysis\\
			\hline
			\hline
			Name&Warning manager bean\\
			\hline
			Input interfaces& Meeting evaluation\\
			\hline
			Output interfaces& Warning generation\\
			\hline
			Description&This component represents the bean which is responsible to generate a warning and submit it to the user if the meeting evaluator notify a conflict. \\
			\hline
				
		\end{tabular}
		
		\caption{Warning manager table } 
		\label{tab:warningmanagertable}
		
	\end{table}
	
\end{flushleft}

\begin{flushleft}
	
	\begin{table}[htp]
		
		\begin{tabular}{l|l}
			Component&analysis\\
			\hline
			\hline
			Name&Conflict solver bean\\
			\hline
			Input interfaces& Warning generation\\
			\hline
			Output interfaces& Conflict resolution\\
			\hline
			Description&The conflict solver is the component that manages the resolution of conflicts, thus it is called by a new warning and his functions are to ask the user whether ignore or modify the meetings involved in a conflict and to apply his choice. \\
			\hline
			
		\end{tabular}
		
		\caption{Conflict solver table } 
		\label{tab:conflictsolvertable}
		
	\end{table}
	
\end{flushleft}

\begin{flushleft}
	
	\begin{table}[htp]
		
		\begin{tabular}{l|l}
			Component&analysis\\
			\hline
			\hline
			Name&Preferences manager bean\\
			\hline
			Input interfaces& Preferences set up \\
			\hline
			Output interfaces& \item Route preferences \item Travel mean preferences \item Break details\\
			\hline
			Description&This is the component designed to assolve all the functions related to the preferences. Hence, it manages the insertion, update and deletion of the preferences about routes, travel means and breaks, furthermore it provides preferences details to other components when it is necessary. \\
			\hline
			
		\end{tabular}
		
		\caption{Preferences manager table } 
		\label{tab:preferencesmanagertable}
		
	\end{table}
	
\end{flushleft}

\begin{flushleft}
	
	\begin{table}[htp]
		
		\begin{tabular}{l|l}
			Component&analysis\\
			\hline
			\hline
			Name&Route calculator bean\\
			\hline
			Input interfaces& \item Travel mean preferences \item Route preferences \item Weather forecasts \item Public transport data \item Maps data \\
			\hline
			Output interfaces& Optimal route\\
			\hline
			Description&This is the component designed to assolve all the functions related to the preferences. Hence, it manages the insertion, update and deletion of the preferences about routes, travel means and breaks, furthermore it provides preferences details to other components when it is necessary. \\
			\hline
			
		\end{tabular}
		
		\caption{Route calculator table } 
		\label{tab:routecalculatortable}
		
	\end{table}
	
\end{flushleft}

\begin{flushleft}
	
	\begin{table}[htp]
		
		\begin{tabular}{l|l}
			Component&analysis\\
			\hline
			\hline
			Name&Signup manager bean\\
			\hline
			Input interfaces& Signup data \\
			\hline
			Output interfaces& Registration\\
			\hline
			Description&This is the component which manages the user signup process, from the insertion of personal data, through their storage on the database, to the account verification. \\
			\hline
			
		\end{tabular}
		
		\caption{Signup manager table } 
		\label{tab:signupmanagertable}
		
	\end{table}
	
\end{flushleft}

\begin{flushleft}
	
	\begin{table}[htp]
		
		\begin{tabular}{l|l}
			Component&analysis\\
			\hline
			\hline
			Name&Login manager bean\\
			\hline
			Input interfaces& Login data \\
			\hline
			Output interfaces& User data \\
			\hline
			Description&This is the component dedicated to all the login steps. When the user insert in the apposite area his name and password, this component takes the data and provides the computations needed to verify the credentials and load the user's content. \\
			\hline
			
		\end{tabular}
		
		\caption{Login manager table } 
		\label{tab:loginmanagertable}
		
	\end{table}
	
\end{flushleft}

